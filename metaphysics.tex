%
% Copyright 2021 Fellipe Augusto Ugliara
%
% Content on this file is licensed under the Creative Commons Attribution 
% 4.0 International License. To view a copy of this license: visit 
% http://creativecommons.org/licenses/by/4.0/, or search for the LICENSE 
% file at https://github.com/ugliara-fellipe/published.works
%

\documentclass[11pt]{article}
\usepackage[a4paper, left=1.11in, right=1.11in, top=1.60in, bottom=1.55in]{geometry}

\usepackage{titling}
\setlength{\droptitle}{-8.5em}

\usepackage[T1]{fontenc}
\usepackage[utf8]{inputenc}

\usepackage{hyphenat}

\usepackage{dirtytalk}
\usepackage{csquotes}

\renewcommand{\thefootnote}{}

\usepackage{lastpage}
\usepackage{fancyhdr}
\pagestyle{fancy}
\fancyhf{}
\fancyhead[L]{\small Random Identifications of Reality \normalsize}
\fancyhead[R]{\small June 20, 2021 \normalsize}
\fancyfoot[L]{\small Copyright 2021 Fellipe Augusto Ugliara. Licensed under the CC BY 4.0 \normalsize}
\fancyfoot[R]{\small \thepage\ of \pageref{LastPage} \normalsize}
\renewcommand{\footrulewidth}{0.5pt}
\fancypagestyle{plain}{
	\renewcommand{\headrulewidth}{0pt}
	\fancyhf{}
	\fancyfoot[L]{\small Copyright 2021 Fellipe Augusto Ugliara. Licensed under the CC BY 4.0 \normalsize}
	\fancyfoot[R]{\small \thepage\ of \pageref{LastPage} \normalsize}}

\title{Random Identifications of Reality}
\author{Fellipe Augusto Ugliara}
\date{June 20, 2021}

\begin{document}
	\maketitle
	
	\section{Introduction} \label{pt-s1}
	
	A priest speaks to the faithful about creation at morning mass. Scholars discuss the composition of matter. The medium communicates with the spirits in meetings bathed by a half light. Monks in the lotus position meditate, seeking Nirvana. An elderly man sitting in the square asks if the birds know what time it is.
	
	The scenarios mentioned are loaded with symbols, theories, stories, perceptions, observations, assumptions about fundamental questions such as: \say{What is being?}, \say{What is the world?}, \say{What is the origin of everything?}, \say{What are thoughts?}, \say{What is time?}.
	
	These questions are present in a variety of contexts; and the answers, attributed to them, constantly end up in some impasse. Investigating these questions and answers, a recurrent feature is observed: by attributing an answer to any of these questions, new examinations can be produced. 
	
	Where does it all come from? If God is a good answer, then \say{What is God's origin?} is a good new question. What is being? If it is a group of atoms, then \say{What is the atom?} will be a new question to consider. It is always possible to examination what came before, and what will happen after. 
	
	Eliminating this recurring aspect requires a new understanding of what the reality would be. It is necessary to break with well-defined ideas, and remove the apparent solidity of the world, to be able to glimpse an eternal sea of randomness.
		
	\section{Potentiality} \label{pt-s2}
	
	It is possible to identify red, yellow, or blue, and each of these options is an identification of something that is also identified as color. It is feasible to identify that something may or may not exist, and these two options are also identified as the concept of existence. Identifying foxes, bears, rabbits, horses, cats is possible and each of these options is also identified as animal, creature.
	
	\textit {Potentiality} is the possibility that something, characteristic, object, creation, idea, concept, thought, memory, feeling, event, whatever it may be, is identified. In the next section, the idea of what would be a \textit {identification} is defined, and how being is interpreted based on this idea. 
	
	\section{Identification} \label{pt-s3}
	
	The elaboration of an answer, regarding what being would be, starts with the following idea: beings, whatever they are, have the potential to be identified. A \textit{identification} is the distinction of a being from other beings. To represent all the potential of identifiable beings, the term \textit{reality} will be used, and to represent an occurrence of the identifications of all these potential identifiable beings, the term \textit{interpretation} will be used.
	
	To understand the essence of identifications, it is reasonable to start by analyzing the amount of identifications that can occur in an interpretation. The occurrence of the following quantities needs to be analyzed: two identifications or more; an identification; no identification.
	
	In interpretations with two or more identifications, if a being is identified in relation to other beings, these will also be identified in relation to the being. No matter which target being identified, the distinction will always be achievable in both directions. It will be possible to distinguish: being from other beings; and other beings, from being.
	
	In interpretations in which only one being is identified, the question arises: Was the being identified in relation to what? It is possible to assume: that there is a medium where the being resides and can be identified, or that there are unidentified beings used as a reference to identify the being.
	
	In the first case, the means of residence can be understood as another being. It is possible to distinguish: the being, from the environment where it resides; and the means of residence, of being. Therefore, the environment where the being resides becomes an identified being. This case then corresponds to the identification of two beings, and not a single being. 
	
	In the second case, the being is identified in relation to the rest of the unidentified beings, who start to be identified in relation to the being. The rest of the unidentified beings become an identified being. Therefore, this case also corresponds to the identification of two beings, and not a single being.
	
	In interpretations that do not exist identifications, the full potential of identifiable beings is still present. In this case, the interpretation corresponds directly with the definition of the reality, and cannot be considered an interpretation, as no identification has occurred.
	
	Therefore, in any interpretation, there are at least two identifications. This fact allows that a definition involving comparison of different identifications can be used in the context of any interpretation. 
	
	Therefore, being is defined as a partial identification of the reality, distinguishable from other identifications that make up the same interpretation. This answer still has gaps open.
	
	In the section \ref{pt-s4} it will be discussed whether the reality can be separated into potential identifiable beings. In the \ref{pt-s5} section, it will be seen how the identifications perceive the interpretation where they occur.
	
	\section{Composition} \label{pt-s4}
	
	To define what is a partial identification of the reality, the concept of how identified beings represent compositions will be explored. An interpretation of what united and separate beings are.
	
	Assuming that one being is made up of two other beings. If they are separated, would the being formed cease to exist? Or conversely, if two united beings form another being, does that other being existing before the separate beings unite? 
	
	Identifying one being in another does not mean that there is any separation between them. This means that the beings of an interpretation can represent greater or lesser identifications of the reality. 
	
	The reality does not change, but depending on how the composition of identifications takes place, different interpretations are produced. An interpretation is a way of representing the reality through a certain composition of identifications.
	
	That not occur are interpretations that have unidentified partials of the reality. By identifying a being, the rest of the reality also becomes identified, as explained in the previous section. 
	
	The idea of composition alone is not enough to clarify how identifications perceive interpretation where they occur. For this, it is also necessary to define how an identification can be represented using the identifications themselves.

	\section{Consistency} \label{pt-s5}
	
	Consistency is a characteristic associated with interpretations. Depending on how the identifications occur, the interpretation will be \textit{consistent} or \textit{inconsistent}. Consistent interpretations will be those perceived by one or some of your identifications.
	
	An identification occurs uniquely in an interpretation. If it can occur in two or more ways, in the same interpretation, it would be an identifiable potential, not an identification.
	
	It is not necessary to specify which identifications need to occur for the interpretation to be consistent. All possible consistent and inconsistent interpretations occur simultaneously. Each of these interpretations represents one of the possible compositions of identifications that can occur.
	
	What characterizes the consistency of the interpretation is the fact that any or some of its identifications are able to perceive this interpretation. Not all interpretations will be consistent, but that doesn't mean they don't exist. These inconsistent interpretations exist, but they are not perceived by their identifications.
	
	An identification is a partial of the reality. It can be said that this identification \textit{perceives} the interpretation in which it occurs: if a smaller part of this partial is a composition of identifications that represents, to a limited extent, the interpretation in which the identification took place.
	
	Limiting \textit{perceptions} follows the same idea as restricting interpretations. Each interpretation is a restricted form of the reality, and each perception is a limited form of interpretation. If interpretation were not a restricted form, it would be the reality itself, and interpretations would not be possible. The same is true with perceptions. If they were not a limited form of interpretation, they would be interpretation itself, and perceptions would not be viable.
	
	Perceptions are also formed by a composition of identifications and this composition is part of the composition of identifications, which forms the interpretation. The identifications of perceptions are not separate or differ from the identifications of interpretations, employing this separation is convenient to define what we know as thoughts, feelings, ideas. We cannot effectively separate perceptions from the reality, they are also compositions of identifications in interpretations of the  reality, but it is convenient to call these compositions of identifications like perceptions.
	
	It is important to emphasize that an inconsistent interpretation can be seen as consistent if only a part of its identifications are considered. This explains how it is possible to develop concepts about interpretations different from those perceived by the identifications, and also explains how it is possible to develop ideas about potentiality, even without being able to fully perceive it.
	
	\section{Ordering} \label{pt-s6}
	
	Ordering is defined that identifications can be arranged one after another, it is to accept that they have some relation of precedence. Ordering is a concept that cannot be applied to the reality, since everything is potential and there are no identified beings to be ordered.
	
	Being able to order is important in interpretations, because if it is not possible to perceive ordering, it would not be feasible to define sequential concepts such as time or counting. Therefore, for some interpretations to be consistent, there are among their identifications those that are perceived in some category of order.
	
	Ordering is similar to composition. A number can be broken down into smaller parts, and we perceive each part as an identification. It can also be joined to another number to form a larger part that we can also perceive through its identification.
	
	Time is similar, it can be perceived as a minute identification, but it can also be separated into two thirty-second parts that are equally perceived identifications. This identification of the two parts of thirty seconds does not really separate the minute, it is only perceived in another way.
	
	Perceiving identifications in some relation of order does not mean that they occur one after another, as all identifications occur simultaneously in the interpretation.
	
	In interpretations where these identifications occur in a way that suggests an ordering, they can make the interpretations consistent. In those that occur in a way that does not suggest an ordering, they can lead to inconsistent interpretations.

	\section{Conclusion} \label{pt-s7}
	
	Interpretations occur in \textit{pulsar} of the reality, the \textit{pulsar} is not a sequential or temporal event, but it happens without ceasing perpetually. The reality pulses continuously, and interpretations occur simultaneously with each pulse. Thus, based on random occurrences of identifications, it is possible to produce consistent interpretations.
	
	Understanding in this way: what being is, what thoughts are, what existence is, overturns the idea that there is a beginning, and an end. Identifying the parts of the potentiality: is to limit the possibilities; it is to outline an interpretation; it is to assume a beginning and an end for each of these parts. It is pure illusion to believe that some set of identifications is the reality itself. An interpretation, one of these sets of identifications, is only a limited representation of the possible representations of potentiality.
	
	\footnote{This work is licensed under the Creative Commons Attribution 4.0 International License. To view a copy of this license, visit http://creativecommons.org/licenses/by/4.0/ or send a letter to Creative Commons, PO Box 1866, Mountain View, CA 94042, USA.}
\end{document}
